% % Список литературы при помощи BibTeX
% Юзать так:
%
% pdflatex rpz
% bibtex rpz
% pdflatex rpz

\bibliographystyle{gost780u}

\begin{thebibliography}{9}
	\bibitem{ftrace} 
	Перехват функций в ядре Linux с помощью ftrace
	\\\texttt{https://m.habr.com/post/413241/}
	
	\bibitem{haifa}
	Haifa Linux Club - Networking Lectures
	\\\texttt{http://haifux.org/network.html}
	
	\bibitem{syscall}
	Loadable Kernel Module Programming and System Call Interception
	\\\texttt{https://www.linuxjournal.com/article/4378}
	
	\bibitem{anatomy} 
	М. Джонс 
	\textit{Анатомия загружаемых модулей ядра Linux}.
	\\\texttt{https://www.ibm.com/developerworks/ru/library/l-lkm/index.html}
	
	\bibitem{sources} 
	Исходные коды ядра Linux
	\\\texttt{http://elixir.free-electrons.com}
\end{thebibliography}


%%% Local Variables: 
%%% mode: latex
%%% TeX-master: "rpz"
%%% End: 
